%===================================== CHAP 3 =================================

\chapter{Research Method}

\section{Litterature Survey Design}

Advantages of doing a survey
- Can be replicated on similar documents or on original documents provided the method is shared and documents are accessible
- Can produce a lot of data at a low cost, in a relatively short time compared to attempting full replications of experiments
- Allows a larger sample population due to the shorter time necessary to evaluate a paper

Disadvantages
- The depth is restricted, does not provide detail on the research topic
- Focuses on what can be counted and measured, other aspects may be overlooked

\subsection{Data requirements}
Want to investigate the reproducibility of experiments published in papers
at AI conferences.

/* Explain variables and relate to best practices!!! */
\begin{description}
\item[Directly topic related]:
        Is source code or data open for the experiment and method?
        Is the method documented?
        Is the experiment documented?
        etc.
\item[Indirectly topic related]
        Research transparency (hypothesis, predictions...)
        Author affiliation (uni/industry/both)
        Novel research?
        Conference view on supplementary material
        Theoretical / Experimental research

\end{description}

Possible analysis patterns
\begin{enumerate}
    \item reproducibility related to author affiliation
    \item reproducibility related to conference view on supplementary material?
    \item reproducibility related to publishing year (improvement over time?)

\end{enumerate}

\subsection{Data generation method}
Documents, conference papers. (Ch. 16)
- Existing conference papers, published openly in the proceedings of the conferences. Physical copies can be ordered, but all accepted papers are available on-line.

Advantages:
- easy to obtain, accessible and are obtained unobtrusively
- allows later longitudinal studies
- other researchers can check and scrutinize the research based on original material

Disadvantages:
-

Sampling frame: accepted papers at IJCAI-13, -16 and AAAI-14 and -16 (can be seen in repo files for sample generation)
Sampling technique: probabilistic random sampling of each conference separately.
        "Probability sampling, as its name suggests, means that the sample has been chosen because the researcher believes that there is a high probability that the sample of respondents (or events) chosen are representative of the overall population being studied. That is, they form a representative cross-section of the overall population." Oates p.96
        Discuss representativeness of sample method.

Sample size: 100 for each conference, restricts the necessary time to conduct the survey while still providing informative accuracy ranges when considering previous research (cite?)
\begin{table}[!h]
\begin{center}
    \begin{tabular}{  l | r  r  r }
    \textbf{Conference} & \textbf{Population Size} & \textbf{Sample Size} & \textbf{Confidence Interval} \\ \hline
    AAAI 2014 & 398 & 100 & 8.49 \\
    AAAI 2016 & 548 & 100 & 8.87 \\
    IJCAI 2013 & 413 & 100 & 8.54 \\
    IJCAI 2016 & 551 & 100 & 8.87 \\ \hline
    Combined & 1910 & 400 & 4.36 \\
    \end{tabular}
\end{center}
\caption{Confidence intervals of survey sample populations given a 50/50  yes/no split with confidence level of 95\%. (\url{https://www.surveysystem.com/sscalc.htm})}
\label{tab:survey-confidence}
\end{table}

\section{Evaluation Procedure}
- Step by step 'instructions'
- Sampling documentation
- Evaluation documentation
- Example evaluations (variable X: "Exhibit A" covers, "Exhibit B" is not enough)

\section{Limitations of the Survey}
- Evaluation bias (modification of variables)
- Sample inconsistency for IJCAI-13 (~50 papers)
- Not an actual attempt at reproducing experiments, researcher's view that discussion of a variable is missing?

\cleardoublepage
