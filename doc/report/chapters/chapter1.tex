%===================================== CHAP 1 =================================

\chapter{Introduction}

\section{Background, Motivation and Problem Outline}
\section{Research Context}
The research was conducted as my Master's thesis at the department of Computer and Information Science at the Norwegian University of Science and Technology. The research task was formulated by Odd Erik Gundersen, my supervisor, and is a continuation of previous work by \citet{3dor.20151059} presented at 3DOR2015\footnote{\url{http://vc.ee.duth.gr/3DOR2015/}}.

\section{Hypothesis, Objectives and Research Questions}
Underlying this thesis is the hypothesis that; \textit{the documentation provided in experimental publications at AI conferences is not good enough to consider the experiments reproducible.}

\begin{description}
    \item[Objective 1] \textit{Evaluate the reproducibility of accepted papers to AI conferences.}
        \begin{description}
            \item[RQ1] What is the state of reproducibility at AI conferences?
        \end{description}
    \item[Objective 2] \textit{Recommend practices that could be adopted to aid the reproducibility of conference papers.}
        \begin{description}
            \item[RQ2] What is generally missing from AI papers to support reproducibility?
            \item[RQ3] What can ease the documentation of missing information from conference papers?
        \end{description}
\end{description}

\section{Research Approach}
\section{Research Contributions}

"We examine the common practices and challenges we
see  in  recent  OSN  research,  from  which  we  propose
a  set  of  recommendations  for  the  benefit  of  OSN
researchers in all disciplines."
\begin{description}
    \item[C1:] \textbf{A survey of experimental research papers from AI conferences.}
    \item[C2:] \textbf{An indication of the state of reproducibility at AI conferences.}
    \item[C3:] \textbf{An approach to measure the reproducibility of AI conference papers.}
\end{description}

\section{Thesis Structure}

\cleardoublepage
